\chapter{Comparaison des affichages}

	\section{Avec labels}
	
\paragraph{}Nous allons à présent comparer le rendu visuel de nos différents générateurs ainsi que de GraphViz qui est de nos jours l'outil le plus utilisé.

\paragraph{TikZ} Nous commençons ce comparatif avec TikZ, qui offre un affichage des labels optimisés.\\

\begin{center}
\resizebox {!}{6cm} {
\begin{tikzpicture}[scale=0.8, every node/.style={scale=0.8}, node distance=1pt]
\node (a3027826572) at (3.500000, -0.000000) {1};
\node (a3027828524) at (3.500000, -1.000000) {2};
\node (a3027828204) at (3.500000, -2.000000) {3};
\node (a3027827948) at (2.000000, -3.000000) {4};
\node (a3027826636) at (1.500000, -4.000000) {7};
\node (a3027826988) at (1.000000, -5.000000) {12};
\node (a3027826604) at (0.500000, -6.000000) {16};
\node (a3027830572) at (0.500000, -7.000000) {18};
\node (a3027830092) at (0.500000, -8.000000) {19};
\node (a3027832332) at (0.000000, -9.000000) {20};
\draw (a3027830092) -- (a3027832332);
\node (a3027832396) at (1.000000, -9.000000) {21};
\draw (a3027830092) -- (a3027832396);
\draw (a3027830572) -- (a3027830092);
\draw (a3027826604) -- (a3027830572);
\draw (a3027826988) -- (a3027826604);
\node (a3027832556) at (1.500000, -6.000000) {17};
\draw (a3027826988) -- (a3027832556);
\draw (a3027826636) -- (a3027826988);
\node (a3027832684) at (2.000000, -5.000000) {13};
\draw (a3027826636) -- (a3027832684);
\draw (a3027827948) -- (a3027826636);
\node (a3027832812) at (2.500000, -4.000000) {8};
\draw (a3027827948) -- (a3027832812);
\draw (a3027828204) -- (a3027827948);
\node (a3027849356) at (3.500000, -3.000000) {5};
\node (a3027849388) at (3.500000, -4.000000) {9};
\node (a3027849420) at (3.000000, -5.000000) {14};
\draw (a3027849388) -- (a3027849420);
\node (a3027849516) at (4.000000, -5.000000) {15};
\draw (a3027849388) -- (a3027849516);
\draw (a3027849356) -- (a3027849388);
\draw (a3027828204) -- (a3027849356);
\node (a3027849676) at (5.000000, -3.000000) {6};
\node (a3027849708) at (4.500000, -4.000000) {10};
\draw (a3027849676) -- (a3027849708);
\node (a3027849804) at (5.500000, -4.000000) {11};
\draw (a3027849676) -- (a3027849804);
\draw (a3027828204) -- (a3027849676);
\draw (a3027828524) -- (a3027828204);
\draw (a3027826572) -- (a3027828524);

\end{tikzpicture}}
\end{center}

\paragraph{Asymptote} Continuons avec le deuxième format \LaTeX de notre application qu'est Asymptote.\\

\begin{center}
\begin{asy}
size(20cm, 20cm);
label("1", (3.500000, -0.000000), E);
label("2", (3.500000, -1.000000), E);
label("3", (3.500000, -2.000000), E);
label("4", (2.000000, -3.000000), E);
label("7", (1.500000, -4.000000), E);
label("12", (1.000000, -5.000000), E);
label("16", (0.500000, -6.000000), E);
label("18", (0.500000, -7.000000), E);
label("19", (0.500000, -8.000000), E);
label("20", (0.000000, -9.000000), E);
draw((0.500000, -8.000000) -- (0.000000, -9.000000));
label("21", (1.000000, -9.000000), E);
draw((0.500000, -8.000000) -- (1.000000, -9.000000));
draw((0.500000, -7.000000) -- (0.500000, -8.000000));
draw((0.500000, -6.000000) -- (0.500000, -7.000000));
draw((1.000000, -5.000000) -- (0.500000, -6.000000));
label("17", (1.500000, -6.000000), E);
draw((1.000000, -5.000000) -- (1.500000, -6.000000));
draw((1.500000, -4.000000) -- (1.000000, -5.000000));
label("13", (2.000000, -5.000000), E);
draw((1.500000, -4.000000) -- (2.000000, -5.000000));
draw((2.000000, -3.000000) -- (1.500000, -4.000000));
label("8", (2.500000, -4.000000), E);
draw((2.000000, -3.000000) -- (2.500000, -4.000000));
draw((3.500000, -2.000000) -- (2.000000, -3.000000));
label("5", (3.500000, -3.000000), E);
label("9", (3.500000, -4.000000), E);
label("14", (3.000000, -5.000000), E);
draw((3.500000, -4.000000) -- (3.000000, -5.000000));
label("15", (4.000000, -5.000000), E);
draw((3.500000, -4.000000) -- (4.000000, -5.000000));
draw((3.500000, -3.000000) -- (3.500000, -4.000000));
draw((3.500000, -2.000000) -- (3.500000, -3.000000));
label("6", (5.000000, -3.000000), E);
label("10", (4.500000, -4.000000), E);
draw((5.000000, -3.000000) -- (4.500000, -4.000000));
label("11", (5.500000, -4.000000), E);
draw((5.000000, -3.000000) -- (5.500000, -4.000000));
draw((3.500000, -2.000000) -- (5.000000, -3.000000));
draw((3.500000, -1.000000) -- (3.500000, -2.000000));
draw((3.500000, -0.000000) -- (3.500000, -1.000000));
\end{asy}

\end{center}

\subparagraph{} L'avantage d'Asymptote réside dans le fait que les labels sont compilés avec \LaTeX. Voici donc un exemple supplémentaire pour montrer les possibilités offerte avec Asymptote.\\

\begin{center}
\begin{asy}[height=8cm, inline=true]
size(20cm, 20cm);
label("$5+\frac{3}{2}$", (0.500000, -0.000000), E);
label("$+$", (0.500000, -1.000000), E);
label("$5$", (0.000000, -2.000000), E);
draw((0.500000, -1.000000) -- (0.000000, -2.000000));
label("$/$", (1.000000, -2.000000), E);
label("$3$", (0.500000, -3.000000), E);
draw((1.000000, -2.000000) -- (0.500000, -3.000000));
label("$2$", (1.500000, -3.000000), E);
draw((1.000000, -2.000000) -- (1.500000, -3.000000));
draw((0.500000, -1.000000) -- (1.000000, -2.000000));
draw((0.500000, -0.000000) -- (0.500000, -1.000000));
\end{asy}

\end{center}

\paragraph{NetworkX} Le dernier rendu, que l'on peut obtenir sous la forme de plusieurs format de fichiers est le suivant:\\

\begin{center}
\includegraphics[width=\columnwidth]{renduNXLabels}
\end{center}
	
\paragraph{GraphViz} Voici pour finir ce comparatif le rendu de GraphViz pour le même arbre.\\

\begin{center}
\includegraphics[height=10cm]{renduGVLabels}
\end{center}
	
	\section{Sans labels}
	
\paragraph{}Le but premier de l'application \verb|TreeDisplay| est de représenter des arbres de grandes tailles. Dans ce cas, les labels deviennent des détails impossibles à lire et donc inutiles. Voilà pourquoi nous effectuons à présent un comparatif sans labels, sur des arbres qui ont plus de n\oe uds que précédemment.

\paragraph{TikZ} Bien que rapidement limité en nombre de n\oe uds comme l'a montré l'étude préliminaire, TikZ permet tout de même de représenter des arbres d'une certaine taille comme ci-dessous.

\begin{center}
\resizebox {!}{0.50\columnwidth} {
\begin{tikzpicture}[scale=0.8, every node/.style={scale=0.8}, node distance=1pt]
\input {renduTikz}
\end{tikzpicture}}
\end{center}

\paragraph{Asymptote} Le visuel obtenu avec Asymptote n'est guère différent de celui de TikZ mais Asymptote a l'avantage de monter beaucoup plus haut en terme de nombre de n\oe uds. Nous n'avons jamais atteint de borne lors de nos essais.

\begin{center}
\input{renduAsy}
\end{center}

\paragraph{NetworkX}

\begin{center}
\includegraphics[width=0.75\columnwidth]{renduNX}
\end{center}
	
\paragraph{GraphViz}

\begin{center}
\includegraphics[height=7cm]{renduGV}
\end{center}