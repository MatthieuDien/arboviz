\chapter{Comparaison des affichages}

	\section{Avec labels}
	
\paragraph{}Nous allons à présent comparer le rendu visuel de nos différents générateurs ainsi que de GraphViz qui est de nos jours l'outil le plus utilisé.
	
\paragraph{GraphViz}

\begin{figure}[h!]
\begin{center}
\includegraphics[width=10cm,height=6cm]{renduGV}
\end{center}
\end{figure}

\paragraph{TikZ}

\begin{figure}[h!] \centering \resizebox {6cm}{10cm} {
\begin{tikzpicture}[scale=0.8, every node/.style={scale=0.8}, node distance=1pt]
\input {renduTikz}
\end{tikzpicture}}
\end{figure}

\paragraph{Asymptote}

\begin{figure}[h!]
\input{renduAsy}
\end{figure}

\paragraph{NetworkX}

\begin{figure}[h!]
\begin{center}
\includegraphics[width=10cm,height=6cm]{renduNX}
\end{center}
\end{figure}
	
	\section{Sans labels}
	
\paragraph{GraphViz}

\begin{figure}[h]
\begin{center}
\includegraphics[width=0.75\columnwidth]{renduGVLabels}
\end{center}
\end{figure}

\paragraph{TikZ}

\begin{figure}[h] \centering \resizebox {!}{0.75\columnwidth} {
\begin{tikzpicture}[scale=0.8, every node/.style={scale=0.8}, node distance=1pt]
\node (a3027826572) at (3.500000, -0.000000) {1};
\node (a3027828524) at (3.500000, -1.000000) {2};
\node (a3027828204) at (3.500000, -2.000000) {3};
\node (a3027827948) at (2.000000, -3.000000) {4};
\node (a3027826636) at (1.500000, -4.000000) {7};
\node (a3027826988) at (1.000000, -5.000000) {12};
\node (a3027826604) at (0.500000, -6.000000) {16};
\node (a3027830572) at (0.500000, -7.000000) {18};
\node (a3027830092) at (0.500000, -8.000000) {19};
\node (a3027832332) at (0.000000, -9.000000) {20};
\draw (a3027830092) -- (a3027832332);
\node (a3027832396) at (1.000000, -9.000000) {21};
\draw (a3027830092) -- (a3027832396);
\draw (a3027830572) -- (a3027830092);
\draw (a3027826604) -- (a3027830572);
\draw (a3027826988) -- (a3027826604);
\node (a3027832556) at (1.500000, -6.000000) {17};
\draw (a3027826988) -- (a3027832556);
\draw (a3027826636) -- (a3027826988);
\node (a3027832684) at (2.000000, -5.000000) {13};
\draw (a3027826636) -- (a3027832684);
\draw (a3027827948) -- (a3027826636);
\node (a3027832812) at (2.500000, -4.000000) {8};
\draw (a3027827948) -- (a3027832812);
\draw (a3027828204) -- (a3027827948);
\node (a3027849356) at (3.500000, -3.000000) {5};
\node (a3027849388) at (3.500000, -4.000000) {9};
\node (a3027849420) at (3.000000, -5.000000) {14};
\draw (a3027849388) -- (a3027849420);
\node (a3027849516) at (4.000000, -5.000000) {15};
\draw (a3027849388) -- (a3027849516);
\draw (a3027849356) -- (a3027849388);
\draw (a3027828204) -- (a3027849356);
\node (a3027849676) at (5.000000, -3.000000) {6};
\node (a3027849708) at (4.500000, -4.000000) {10};
\draw (a3027849676) -- (a3027849708);
\node (a3027849804) at (5.500000, -4.000000) {11};
\draw (a3027849676) -- (a3027849804);
\draw (a3027828204) -- (a3027849676);
\draw (a3027828524) -- (a3027828204);
\draw (a3027826572) -- (a3027828524);

\end{tikzpicture}}
\end{figure}

\paragraph{Asymptote}

\begin{figure}[h]
\begin{asy}
size(20cm, 20cm);
label("1", (3.500000, -0.000000), E);
label("2", (3.500000, -1.000000), E);
label("3", (3.500000, -2.000000), E);
label("4", (2.000000, -3.000000), E);
label("7", (1.500000, -4.000000), E);
label("12", (1.000000, -5.000000), E);
label("16", (0.500000, -6.000000), E);
label("18", (0.500000, -7.000000), E);
label("19", (0.500000, -8.000000), E);
label("20", (0.000000, -9.000000), E);
draw((0.500000, -8.000000) -- (0.000000, -9.000000));
label("21", (1.000000, -9.000000), E);
draw((0.500000, -8.000000) -- (1.000000, -9.000000));
draw((0.500000, -7.000000) -- (0.500000, -8.000000));
draw((0.500000, -6.000000) -- (0.500000, -7.000000));
draw((1.000000, -5.000000) -- (0.500000, -6.000000));
label("17", (1.500000, -6.000000), E);
draw((1.000000, -5.000000) -- (1.500000, -6.000000));
draw((1.500000, -4.000000) -- (1.000000, -5.000000));
label("13", (2.000000, -5.000000), E);
draw((1.500000, -4.000000) -- (2.000000, -5.000000));
draw((2.000000, -3.000000) -- (1.500000, -4.000000));
label("8", (2.500000, -4.000000), E);
draw((2.000000, -3.000000) -- (2.500000, -4.000000));
draw((3.500000, -2.000000) -- (2.000000, -3.000000));
label("5", (3.500000, -3.000000), E);
label("9", (3.500000, -4.000000), E);
label("14", (3.000000, -5.000000), E);
draw((3.500000, -4.000000) -- (3.000000, -5.000000));
label("15", (4.000000, -5.000000), E);
draw((3.500000, -4.000000) -- (4.000000, -5.000000));
draw((3.500000, -3.000000) -- (3.500000, -4.000000));
draw((3.500000, -2.000000) -- (3.500000, -3.000000));
label("6", (5.000000, -3.000000), E);
label("10", (4.500000, -4.000000), E);
draw((5.000000, -3.000000) -- (4.500000, -4.000000));
label("11", (5.500000, -4.000000), E);
draw((5.000000, -3.000000) -- (5.500000, -4.000000));
draw((3.500000, -2.000000) -- (5.000000, -3.000000));
draw((3.500000, -1.000000) -- (3.500000, -2.000000));
draw((3.500000, -0.000000) -- (3.500000, -1.000000));
\end{asy}

\end{figure}

\paragraph{NetworkX}

\begin{figure}[h]
\begin{center}
\includegraphics[width=0.75\columnwidth]{renduNXLabels}
\end{center}
\end{figure}